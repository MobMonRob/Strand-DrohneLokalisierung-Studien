\chapter{Theoretische Grundlagen}


\section{Sensoren}

    \subsection{Magnetometer}

    \subsection{Gyroskyp}

    \subsection{Laserscanner}

    \subsection{Kamera}

        \subsubsection{Bild Kamera}

        \subsubsection{Infrarot Kamera}

    \subsection{Abstandssensoren}


\section{ROS}

ROS (Robot Operating System) ist eine Open-Source-Plattform, die speziell für die Entwicklung von Robotersoftware entwickelt wurde. Es bietet eine Reihe von Bibliotheken, Tools und Frameworks, die es Entwicklern ermöglichen, komplexe Robotikanwendungen zu erstellen und zu betreiben. ROS wurde von Willow Garage entwickelt und ist heute ein weit verbreitetes Framework in der Robotik-Community.

ROS besteht aus verschiedenen Modulen, die es ermöglichen, Roboterhardware und -software zu abstrahieren und zu standardisieren. Die Plattform bietet eine Vielzahl von Werkzeugen für die Entwicklung von Robotik-Software, einschließlich Visualisierungstools, Datenverarbeitungs- und Analysetools sowie eine umfassende Dokumentation.

ROS ist so konzipiert, dass es auf einer Vielzahl von Betriebssystemen und Hardwarearchitekturen laufen kann und bietet Unterstützung für eine breite Palette von Robotern und Sensoren. Es ist auch bekannt für seine Fähigkeit zur Zusammenarbeit zwischen verschiedenen Robotern, die miteinander kommunizieren und Aufgaben gemeinsam erledigen können.

Dank seiner leistungsstarken Funktionen und Flexibilität ist ROS zu einem der wichtigsten Frameworks für die Robotik-Entwicklung geworden und wird in vielen Anwendungen eingesetzt, von industriellen Robotern bis hin zu autonomen Fahrzeugen.



Regenerate response

    \subsection{Nodes}

    \subsection{Topics}

    \subsection{Publish and Subscripe Pattern}

    \subsection{Objekterkennung}

    \subsection{QR-Codes}

\section{Drohne/Multicopter}

\section{3D-Modelle}

    \subsection{3D-Scanning}

\section{Positionsbestimmung}

    \subsection{Spatial Mapping}

    \subsection{Inertielle Positionsbestimmung}

\section{Regelsysteme}

    \subsection{PID-Regler}

\section{Simulationstechnik}

\section{Problembehebung}
