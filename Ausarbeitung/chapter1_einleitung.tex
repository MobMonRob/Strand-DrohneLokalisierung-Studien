\chapter{Einleitung}

GPS Signale sucht man in Räumen vergeblich. Das GPS System wurde nicht dafür konzipiert in Räumen zu funktionieren.
In dieser Studienarbeit geht es darum zu ermitteln auf welche Art und Weise eine Drohne in einem 3D Modell ohne GPS navigieren kann.

3D Navigation ist ein wichtiger Aspekt bei der Entwicklung von Systemen, die in einer virtuellen oder realen Umgebung agieren. Die Nutzung eines externen Umweltmodells bietet hierbei viele Vorteile, da es die Navigation vereinfacht und gleichzeitig die Genauigkeit verbessert. Ein externes Umweltmodell kann dabei in Form einer digitalen Karte oder eines virtuellen Raums im Computer vorliegen. In diesem Zusammenhang wird die 3D Navigation anhand eines externen Umweltmodells untersucht und die Vorteile, die sich daraus ergeben, werden dargestellt.



\section{Ausgangssituation}

Prinzipiell hat eine Drohne verschiedene Möglichkeiten eine Positionsbestimmung durchzuführen.
Eine Drohne kann mithilfe von Kameras und inertialen Sensoren die Position relativ zu einer Startposition bestimmen, oder \ac{GPS} verwenden um die Position mithilfe von Satelliten zu möglichst genau zu bestimmen.
Schnell hat man festgestellt, dass die genaue Positionsbestimmmung über GPS in Gebäuden nicht funktioniert, da die Signale der Satelliten zu schwach sind. Außerdem bietet \ac{GPS} ohne großen Aufwand keine Möglichkeiten eine Genauigkeit von unter zwei Metern bekommen
.




