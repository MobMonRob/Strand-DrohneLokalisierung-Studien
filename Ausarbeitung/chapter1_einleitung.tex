\chapter{Einleitung}

GPS Signale sucht man in Räumen vergeblich. Das GPS System wurde nicht dafür konzipiert in Räumen zu funktionieren.
In dieser Studienarbeit geht es darum zu ermitteln auf welche Art und Weise eine Drohne in einem 3D Modell ohne GPS navigieren kann.
3D Navigation ist ein wichtiger Aspekt bei der Entwicklung von Systemen, die in einer virtuellen oder realen Umgebung agieren. Die Nutzung eines externen Umweltmodells bietet hierbei viele Vorteile, da es die Navigation vereinfacht und gleichzeitig die Genauigkeit verbessert. Ein externes Umweltmodell kann dabei in Form einer digitalen Karte oder eines virtuellen Raums im Computer vorliegen. In diesem Zusammenhang wird die 3D Navigation anhand eines externen Umweltmodells untersucht und die Vorteile, die sich daraus ergeben, werden dargestellt.



\section{Ausgangssituation}

Prinzipiell hat eine Drohne verschiedene Möglichkeiten eine Positionsbestimmung durchzuführen.
Eine Drohne kann mithilfe von Kameras oder inertialen Sensoren wie Gyroskopen und Beschleunigungssensoren die Position relativ zu einer Startposition bestimmen, oder \ac{GPS} verwenden um die Position mithilfe von Satelliten möglichst genau zu bestimmen.
Schnell hat man festgestellt, dass die genaue Positionsbestimmung über GPS in Gebäuden nicht funktioniert, da die Signale der Satelliten zu schwach sind. Außerdem bietet \ac{GPS} ohne großen Aufwand keine Möglichkeiten eine genaue Bestimmung von unter zwei Metern zu gewährleisten.
Um eine Navigation ohne GPS zu tätigen muss zudem ein 3D Modell der Umgebung der Drohne übermittelt werden.
Zu Beginn muss das 3D Modell in dem die Drohne sich bewegt, noch erstellt werden.


\section{Motivation}

%Die Navigation im dreidimensionalen Raum ist ein wichtiger Schritt für die Automatisierung und Autonomisierung verschiedener Prozesse. Beispielsweise im Bereich der Aufklärung und der Logistik.

%Die Navigation im dreidimensionalen Raum in Gebäuden ist ein wichtiger Schritt um in vielen Gebieten in Gebäuden ohne menschliche Truppen eine Sichtung der Umgebung zu ermöglichen. Beispielsweise könnte die Feuerwehr bei einem Wohnhausbrand mit einer Drohne nach Leuten suchen, die noch im Gebäude feststecken.

%Diese Arbeit dient dazu Grundlagen zur Navigation mithilfe eines 3D Modells der Umgebung in einem dreidimensionalen Raum zu erarbeiten. Diese Arbeit kann danach in weiteren Arbeiten verwendet werden, um die Navigation in erweiterten Teilbereichen einzusetzen und sie um weitere Funktionen zu ergänzen.

In der heutigen Zeit wird das Navigieren in komplexen Umgebungen immer wichtiger und herausfordernder. Egal, ob in der Luftfahrt, bei der Navigation von Drohnen oder autonomen Fahrzeugen - die Fähigkeit, präzise und schnell in dreidimensionalen Räumen zu navigieren, ist unverzichtbar geworden.

Um diese Herausforderungen zu meistern kann ein externes Umweltmodell verwendet werden. Es ermöglicht eine präzisere und realitätsnähere Darstellung der Umgebung und kann somit eine genauere Navigation ermöglichen. Durch die Verwendung von 3D-Modellen kann man beispielsweise Hindernisse in der Umgebung besser identifizieren und umfahren, ohne dass es zu Kollisionen kommt. Auch die Planung von Routen und die Optimierung von Fahrzeugbewegungen können durch ein externes Umweltmodel verbessert werden. Zudem wird durch das Verwenden eines 3D-Modells die Navigation innerhalb schwieriger Umgebungen erleichtern. Der Vorteil hiervon ist, dass man nun gefährliche Gebiete deutlich einfacher erkunden kann. Aber auch in Bereichen wie der Architektur, der Planung von Fabrikanlagen oder der Simulation von Gefahrensituationen kann die Verwendung von 3D-Modellen für eine effiziente Navigation von großer Bedeutung sein.

Insgesamt bietet die 3D-Navigation mit Hilfe von externen Umweltmodellen also zahlreiche Vorteile und kann in vielen verschiedenen Anwendungsbereichen von Nutzen sein. Sei es in der Logistik, der Aufklährung oder der Suche nach z.B. Personen in Gefahrenumgebungen.

Diese Arbeit dient dazu Grundlagen zur Navigation mithilfe eines 3D Modells der Umgebung in einem dreidimensionalen Raum zu erarbeiten. Diese Arbeit kann danach in weiteren Arbeiten verwendet werden, um die Navigation in erweiterten Teilbereichen einzusetzen und sie um weitere Funktionen zu ergänzen.

