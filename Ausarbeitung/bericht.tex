%%%%%%%%%%%%%%%%%%%%%%%%%%%%%%%%%%%%%%%%%%%%%%%%%%%%%%%%%%%%%%%%%%%%%%%%%%%%%%%
%% Descr:       Vorlage für Berichte der DHBW-Karlsruhe
%% Author:      Prof. Dr. Jürgen Vollmer, vollmer@dhbw-karlsruhe.de
%% $Id: bericht.tex,v 1.19 2016/03/16 16:59:41 vollmer Exp $
%%  -*- coding: utf-8 -*-
%%%%%%%%%%%%%%%%%%%%%%%%%%%%%%%%%%%%%%%%%%%%%%%%%%%%%%%%%%%%%%%%%%%%%%%%%%%%%%%

\documentclass[
   ngerman          % neue deutsche Rechtschreibung
  ,a4paper          % Papiergrösse
% ,twoside          % Zweiseitiger Druck (rechts/links)
% ,10pt             % Schriftgrösse
% ,11pt
  ,12pt
  ,pdftex
 %,disable         % Todo-Markierungen auschalten
]{report}

% Bitte die Codierung Ihrer Dateien auswählen:
% \usepackage[latin1]{inputenc}    % Für UNIX mit ISO-LATIN-codierten Dateien
% \usepackage[applemac]{inputenc}  % Für Apple Mac
% \usepackage[ansinew]{inputenc}   % Für Microsoft Windows
\usepackage[utf8]{inputenc}        % UTF-8 codierte Dateien
                                   % Dieses Dokument ist unter Unix erstellt, daher
                                   % wird diese Input-Codierung benutzt.

\usepackage{bericht}

%%%%%%%%%%%%%%%%%%%%%%%%%%%%%%%%%%%%%%%%%%%%%%%%%%%%%%%%%%%%%%%%%%%%%%%%%%%%%%%
%% Angaben zur Arbeit
%%%%%%%%%%%%%%%%%%%%%%%%%%%%%%%%%%%%%%%%%%%%%%%%%%%%%%%%%%%%%%%%%%%%%%%%%%%%%%%

\newcommand{\Autor}{Maximilian Burr, Johannes Vater, Fabian Droll}
\newcommand{\MatrikelNummerBurr}{Burr: XXXXXXXXX}
\newcommand{\MatrikelNummerVater}{Vater: XXXXXXXXX}
\newcommand{\MatrikelNummerDroll}{Droll: XXXXXXXXX}
\newcommand{\Kursbezeichnung}{TINF20B3}

\newcommand{\FirmenLogoDeckblatt}{\fbox{\includegraphics[width=3cm]{lion}}}

% Falls es kein Firmenlogo gibt:
%  \newcommand{\FirmenLogoDeckblatt}{}

\newcommand{\BetreuerDHBW}{Marcus Strand}

%%%%%%%%%%%%%%%%%%%%%%%%%%%%%%%%%%%%%%%%%%%%%%%%%%%%%%%%%%%%%%%%%%%%%%%%%%%%%%%%%%%%%

\newcommand{\Was}{Studienarbeit}
% Wird auf dem Deckblatt in der Erklärung benutzt

%%%%%%%%%%%%%%%%%%%%%%%%%%%%%%%%%%%%%%%%%%%%%%%%%%%%%%%%%%%%%%%%%%%%%%%%%%%%%%%%%%%%%

\newcommand{\Titel}{3D Navigation anhand eines externen Umweltmodels}
\newcommand{\AbgabeDatum}{22. Mai 2023}

\newcommand{\Dauer}{XX Wochen}

\newcommand{\Abschluss}{Bachelor of Engineering}
% \newcommand{\Abschluss}{Bachelor of Science}

\newcommand{\Studiengang}{Informationstechnik}
% \newcommand{\Studiengang}{Angewandte Informatik}

\hypersetup{%%
  pdfauthor={\Autor},
  pdftitle={\Titel},
  pdfsubject={\Was}
}

%%%%%%%%%%%%%%%%%%%%%%%%%%%%%%%%%%%%%%%%%%%%%%%%%%%%%%%%%%%%%%%%%%%%%%%%%%%%%%%

% Wenn \includeonly{..} benutzt wird, werden nur diese Kaptitel ausgegeben.
\includeonly{
  abk
  ,chapter1_einleitung
  ,chapter2_problemstellung
  ,chapter3_theoretische_grundlagen
  ,chapter4_probleme
  ,chapter5_hardware
  ,chapter6_software
  ,chapter6_fazit
 ,changelog
}

%%%%%%%%%%%%%%%%%%%%%%%%%%%%%%%%%%%%%%%%%%%%%%%%%%%%%%%%%%%%%%%%%%%%%%%%%%%%%%%

% Benutzt man das "biblatex"-Paket, dann muß das hier stehen:
% siehe auch die mit BIBLATEX markierten Zeilen in bericht.sty
\bibliography{bericht}

\begin{document}

%%%%%%%%%%%%%%%%%%%%%%%%%%%%%%%%%%%%%%%%%%%%%%%%%%%%%%%%%%%%%%%%%%%%%%%%%%%%%%%

\begin{titlepage}
\begin{center}
\vspace*{-2cm}
\FirmenLogoDeckblatt\hfill\includegraphics[width=4cm]{dhbw-logo}\\[2cm]
{\Huge \Titel}\\[2cm]
{\Huge\scshape \Was}\\[2cm]
{\large für die Prüfung zum}\\[0.5cm]
{\Large \Abschluss}\\[0.5cm]
{\large des Studienganges \Studiengang}\\[0.5cm]
{\large an der}\\[0.5cm]
{\large Dualen Hochschule Baden-Württemberg Karlsruhe}\\[0.5cm]
{\large von}\\[0.5cm]
{\large\bfseries \Autor}\\[1cm]
{\large Abgabedatum \AbgabeDatum}
\vfill
\end{center}
\begin{tabular}{l@{\hspace{2cm}}l}
Bearbeitungszeitraum	         & \Dauer 			\\
Matrikelnummern	                 & \MatrikelNummerBurr		\\
               & \MatrikelNummerVater   \\
               & \MatrikelNummerDroll   \\
Kurs			         & \Kursbezeichnung		\\
Gutachter der Studienarbeit	 & \BetreuerDHBW		\\
\end{tabular}
\end{titlepage}

%%%%%%%%%%%%%%%%%%%%%%%%%%%%%%%%%%%%%%%%%%%%%%%%%%%%%%%%%%%%%%%%%%%%%%%%%%%%%%%

% Nur für Bachelorarbeiten einfügen:
\input{erklaerung.tex}

\begin{abstract}
Dieses \LaTeX-Dokument kann als Vorlage für einen Praxis- oder Projektbericht, eine Studien- oder
Bachelorarbeit dienen.

Zusammengestellt von Prof. Dr. Jürgen Vollmer \email{vollmer@dhbw-karlsruhe.de}\\
\url{http://www.dhbw-karlsruhe.de}

\centering Stand \verb+$Date: 2016/03/16 16:59:41 $+
\end{abstract}

\newpage
\tableofcontents           % Inhaltsverzeichnis hier ausgeben
\listoffigures             % Liste der Abbildungen
\listoftables              % Liste der Tabellen
\lstlistoflistings         % Liste der Listings
\listofequations           % Liste der Formeln

% Jetzt kommt der "eigentliche" Text
%%%%%%%%%%%%%%%%%%%%%%%%%%%%%%%%%%%%%%%%%%%%%%%%%%%%%%%%%%%%%%%%%%%%%%%%%%%%%%
%% Descr:       Vorlage für Berichte der DHBW-Karlsruhe, Datei mit Abkürzungen
%% Author:      Prof. Dr. Jürgen Vollmer, vollmer@dhbw-karlsruhe.de
%% $Id: abk.tex,v 1.3 2016/03/16 12:21:40 vollmer draft $
%% -*- coding: utf-8 -*-
%%%%%%%%%%%%%%%%%%%%%%%%%%%%%%%%%%%%%%%%%%%%%%%%%%%%%%%%%%%%%%%%%%%%%%%%%%%%%%%

\chapter*{Abkürzungsverzeichnis}                   % chapter*{..} -->   keine Nummer, kein "Kapitel"
						         % Nicht ins Inhaltsverzeichnis
% \addcontentsline{toc}{chapter}{Akürzungsverzeichnis}   % Damit das doch ins Inhaltsverzeichnis kommt

% Hier werden die Abkürzungen definiert
\begin{acronym}[DHBW]
  % \acro{Name}{Darstellung der Abkürzung}{Langform der Abkürzung}
 \acro{Abk}[Abk.]{Abkürzung}

 \acro{GPS}{Global Positioning System}
 \acro{SLAM}{Simulaneous Localization and Mapping}
 \acro{RGB-D}{RGB-Color-Depth}
 \acro{PID}{Proportional-Integral-Differential}
 \acro{IMU}{Inertial Measurement Unit}
 \acro{ToF}{Time of Flight}
 \acro{OS}{Operating System}
 \acro{ROS}{Robot Operating System}
 \acro{MAVLink}{Micro Air Vehicle Link}
 \acro{USB}{Universal Serial Bus}
 \acro{UDP}{User Datagram Protocol}
 \acro{TCP}{Transmission Control Protocol}
 \acro{UAV}{Unmanned Aerial Vehicle}
 \acro{P-Regler}{Proportional Regler}
 \acro{I-Regler}{Integral Regler}
 \acro{D-Regler}{Differential-Regler}
 \acro{EKF}{Extended Kalman Filter}
 \acro{VSLAM}{Visual SLAM}
 \acro{VO}{Visuelle Odometrie}
 \acro{VDI}{Verein Deutscher Ingenieure}
 \acro{SIL}{Software-in-the-loop}
 \acro{HIL}{Hardware-in-the-loop}
 \acro{OBBs}{Oriented Bounding Boxen}
 \acro{Lidar}{Light Detection and Ranging}
 \acro{AR}{Augmented Reality}
 \acro{DK}{Development Kit}
 \acro{TF}{Transform Library}
 \acro{TUM}{Technische Universität München}

\end{acronym}
              % Abkürzungsverzeichnis
%\include{einleitung}
%\include{problemstellung}
%\chapter{Theoretische Grundlagen}

\section{Regelsysteme}

\section{Hololens}

\section{Regelsysteme}

\section{Spatial Mapping}

\chapter{Einleitung}

GPS Signale sucht man in Räumen vergeblich. Das GPS System wurde nicht dafür konzipiert in Räumen zu funktionieren.
In dieser Studienarbeit geht es darum zu ermitteln auf welche Art und Weise eine Drohne in einem 3D Modell ohne GPS navigieren kann.

\section{Ausgangssituation}

Eine Drohne kann mithilfe von Kameras und Inertialen Sensoren die Position relativ zu einer Startposition bestimmen.



\chapter{Problemstellung}

Die Navigation anhand eines 3D Modell bringt gewisse theoretische Probleme mit sich.

\section{Erwartete Probleme} \label{erwartete_probleme:section}

In der Vorbereitung der Arbeit wurden bereits Probleme erkannt, die auf jeden Fall gelöst oder Alternativen dazu gefunden werden müssen, um die erfolgreiche Umsetzung des Projekts zu ermöglichen. Im Folgendem werden diese Probleme erläutert.

\begin{description}
    \item[Sensordrift] Die inertialen Sensoren der Drohne funktionieren über Integrationsfunktionen. Die Messfehler in den inertialen Sensoren können im Laufe der Zeit akkumulieren und die Genauigkeit der Positions- und Bewegungsinformationen der Drohne beeinträchtigen. Dieses Phänomen nennt man Sensordrift.
    \item[Rechenleistung] Um die \ac{SLAM} Algorithmen zu verwenden wird ohne eine effiziente Implementierung sehr viel Rechenleistung benötigt.
    \item[Kalibrierung] Um die Drohne stabil in der Luft zu halten ist eine stabile Kontrolle der Rotoren notwendig. Dazu muss der \ac{PID} Regler auf der Drohne gut kalibriert sein.
\end{description}

\subsection{Lokalisierung der Drohne} \label{lokalisierung_der_drohne:subsection}

Von vorneherein war klar, dass die Lokalisierung der Drohne nicht über \ac{GPS} getätigt werden kann. Für die Aufgaben, welche die Drohne übernehmen soll, ist eine bestmögliche Positionserkennung vonnöten. Die Genauigkeit mittels \ac{GPS} unter den technischen Bedingungen der Drohne beträgt nur zwei Meter. Das weitaus größer Problem mit \ac{GPS} ist allerdings, dass man innerhalb von Gebäude keine Verbindung zu den Satelliten herstellen kann. Aufgrund dieser Probleme ist eine Verwendung von \ac{GPS} zur Lokalisierung der Drohne nicht möglich. Dazu mussten alternative Wege gefunden werden, um eine genaue Lokalisierung ohne \ac{GPS} in Gebäuden zu ermöglichen. Es werden andere, wie in Auflistung \ref{lst:navigation-types} beschriebene Verfahren benötigt.

\begin{description} \label{lst:navigation-types}
    \item [\textbf{Inertielle Navigation}] Die Trägheitsnavigation beruht auf der Verwendung von Trägheitssensoren wie Beschleunigungsmessern und Gyroskopen zur Erfassung der Bewegungen und der Ausrichtung der Drohne. Durch Integration der gemessenen Beschleunigungs- und Drehratenwerte über die Zeit kann die aktuelle Position und Ausrichtung der Drohne bestimmt werden. Die Trägheitsnavigation ist jedoch anfällig für Fehler, die sich mit der Zeit akkumulieren und zu immer größeren Ungenauigkeiten führen. Um die Genauigkeit zu verbessern, können Trägheitsnavigationssysteme mit anderen Lokalisierungstechnologien kombiniert werden, z. B. mit optischen Sensoren oder Magnetometern.

    \item [\textbf{Optische Navigation}] Optische Sensoren, wie Kameras oder Tiefenkameras, können verwendet werden, um die Position und Bewegung der Drohne zu erkennen. Durch die Analyse von Bildern oder Tiefeninformationen aus der Umgebung kann die Drohne ihre Position relativ zur Umgebung bestimmen. Diese Methode erfordert jedoch eine geeignete Umgebung mit ausreichend markanten Merkmalen oder Strukturen, die von den optischen Sensoren erkannt werden können. Zudem kann die Genauigkeit der optischen Lokalisierung durch Beleuchtungsveränderungen oder visuelle Ähnlichkeiten zwischen verschiedenen Bereichen beeinträchtigt werden.

    \item [\textbf{Ultraschallnavigation}] Ultraschall- oder Infrarotsensoren können eingesetzt werden, um die Abstände zu Wänden, Hindernissen oder anderen festen Strukturen in der Umgebung zu messen. Durch die Kombination der gemessenen Abstandswerte mit den Bewegungsdaten der Drohne können Position und Bewegung verfolgt werden. Diese Methode eignet sich insbesondere für den Einsatz in Innenräumen, da Ultraschall- oder Infrarotsignale durch Wände und andere Hindernisse hindurchgehen können. Allerdings sind diese Sensoren anfällig für Reflexionen und Mehrdeutigkeiten, die zu Fehlern in der Lokalisierung führen können.

    \item [\textbf{Magnetfeldnavigation}] Magnetfeldsensoren können genutzt werden, um das Magnetfeld der Erde zu messen und dadurch die Ausrichtung der Drohne zu bestimmen. Durch den Vergleich der gemessenen Magnetfeldwerte mit bekannten Magnetfeldkarten kann die Drohne ihre Position relativ zur Umgebung bestimmen. Diese Methode ist besonders nützlich in Innenräumen, wo \ac{GPS} nicht verfügbar ist. Allerdings kann die Genauigkeit der magnetfeldbasierten Lokalisierung durch Magnetfeldstörungen in der Umgebung, verursacht durch metallische Objekte oder elektrische Geräte, beeinträchtigt werden.

    \item [\textbf{Funkbasierte Lokalisierung:}] Funkbasierte Lokalisierungstechnologien wie Wi-Fi, Bluetooth oder Ultra-Wideband (UWB) können verwendet werden, um die Position der Drohne zu bestimmen. Durch den Empfang und dieAuswertung von Signalstärkeinformationen mehrerer Funkquellen in der Umgebung kann die Drohne ihre Position triangulieren. Diese Methode erfordert jedoch eine ausreichende Abdeckung mit Funkquellen und eine präzise Signalstärkemessung, um genaue Lokalisierungsergebnisse zu erzielen.
\end{description}

Die verwendete COEX Clover Drohne (siehe Kapitel \ref{drohne:section}) verfügt standardmäßig über die in \ref{lst:coex-sensoren} aufgeführten Sensoren.

\begin{center}
\begin{itemize}
    \item Beschleunigungssensoren 
    \item Gyroskope
    \item Magnetometer
    \item Videokamera
    \item GPS
    \item Abstandssensor
\end{itemize}
\label{lst:coex-sensoren}
\end{center}

\subsection{Erfassung des 3D Modells der Karte} \label{erfassung_des_3d-modells:subsection}

Für die Erfassung des 3D-Modells gibt es im Wesentlichen zwei verschiedene Ansätze: vorgescannte Karten und \ac{SLAM}.

Beim ersten Ansatz erfolgt die Navigation innerhalb einer vordefinierten Karte, die dem System bereits bekannt ist. In diesem Fall wurde das 3D-Modell der Umgebung bereits vor dem Einsatz der Drohne erstellt. Die zuvor gescannte Karte dient als Referenz für die Lokalisierung und Navigation der Drohne. Die Drohne kann verschiedene Lokalisierungstechniken, wie die im vorherigen Abschnitt erwähnten, verwenden, um ihre Position innerhalb der vorab gescannten Karte zu bestimmen und entsprechend zu navigieren. Dieser Ansatz wird in der Regel in Szenarien verwendet, in denen die Umgebung relativ statisch ist und die Karte im Voraus mit speziellen Geräten oder Techniken erstellt werden kann.

Der zweite Ansatz besteht darin, die Karte dynamisch in Echtzeit zu erstellen, während die Drohne ihre Umgebung erkundet. Hier kommt SLAM ins Spiel. SLAM ist eine Technik, die es der Drohne ermöglicht, gleichzeitig die Umgebung zu kartieren und sich selbst in dieser Karte zu lokalisieren. Die Drohne nutzt ihre Sensoren, z.B. Kameras oder Tiefensensoren, um Daten über die Umgebung zu sammeln, während sie sich bewegt. Anschließend verarbeitet sie diese Daten mit SLAM-Algorithmen, um ein 3D-Modell der Umgebung zu erstellen und ihre eigene Position in diesem Modell zu bestimmen. SLAM ermöglicht es der Drohne, in unbekannten oder sich verändernden Umgebungen zu navigieren, in denen zuvor gescannte Karten nicht verfügbar oder nicht aktuell sind.

Beide Ansätze haben ihre Vorteile, aber auch ihre Grenzen. Vorab gescannte Karten stellen eine bekannte und zuverlässige Referenz für die Lokalisierung und Navigation dar, erfordern jedoch eine vorherige Kenntnis der Umgebung. Auf der anderen Seite ermöglicht SLAM den Einsatz der Drohne in unbekannten oder sich verändernden Umgebungen, kann aber rechenintensiver und fehleranfälliger sein. Die Wahl des Ansatzes hängt von den spezifischen Anforderungen der Anwendung und den Merkmalen der Betriebsumgebung ab.



\subsection{Nutzung der Software} \label{nutzung_der_software:subsection}

Um die Clover-Drohne effektiv einsetzen zu können, sind mehrere Softwarekomponenten erforderlich. Dazu gehören:

\begin{description}
    \item[\textbf{PX4-Autopilot}] Die PX4-Software ist ein Open-Source-Flugsteuerungssystem, das die Steuerung und Navigation der Drohnenhardware übernimmt. Sie bietet Funktionen wie Flugsteuerung, Stabilisierung, Missionsplanung und Kommunikation mit den Sensoren und Aktoren der Drohne. Die PX4-Software kann auf der Clover-Drohne installiert und konfiguriert werden, um eine präzise Steuerung und Flugleistung zu gewährleisten.
    
    \item[\textbf{Robot Operating System}] ROS ist ein flexibles Framework für die Entwicklung von Robotersoftware. Es bietet eine Vielzahl von Tools, Bibliotheken und Paketen, die für die Entwicklung des Navigationssystems der Clover-Drohne genutzt werden können. ROS ermöglicht die Integration verschiedener Sensoren, die Verarbeitung von Sensorinformationen, die Implementierung von Lokalisierungs- und Kartierungsalgorithmen sowie die Steuerung der Drohne. Die Clover-Drohne kann über eine ROS-Schnittstelle mit der PX4-Software verbunden werden, um eine nahtlose Kommunikation zwischen den Steuerungs- und Navigationssystemen zu ermöglichen.
    \item[\textbf{\ac{SLAM}}] Zur Erstellung und Aktualisierung des Echtzeit-3D-Modells der Umgebung wird ein \ac{SLAM} Verfahren eingesetzt. \ac{SLAM} ermöglicht es der Drohne, ihre Position innerhalb der Umgebung zu bestimmen und gleichzeitig während des Fluges eine Karte der Umgebung zu erstellen. Es gibt verschiedene SLAM-Algorithmen und -Implementierungen, die je nach den Anforderungen und Ressourcen der Clover-Drohne ausgewählt werden können. Einige beliebte SLAM-Bibliotheken sind GMapping, Cartographer und RTAB-Map.
    
    \item[\textbf{Entwicklungsumgebung}]  Um mit ROS zu entwickeln und die Drohne zu simulieren, ist eine geeignete Entwicklungsumgebung erforderlich. Eine beliebte Wahl ist die Verwendung von "Visual Studio Code" zusammen mit den ROS-Erweiterungen. Diese ermöglichen eine effiziente Entwicklung, Fehlersuche und Überwachung von ROS-Programmen. Darüber hinaus können Simulationswerkzeuge wie Gazebo eingesetzt werden, um die Drohne und ihre Umgebung virtuell zu simulieren und verschiedene Szenarien zu testen, bevor sie auf der eigentlichen Hardware ausgeführt werden.

    Durch den Einsatz dieser Softwarekomponenten und geeigneter Entwicklungswerkzeuge kann das volle Potenzial der Clover-Drohne ausgeschöpft werden. Von der Flugsteuerung und Navigation bis hin zur Kartenerstellung und -aktualisierung ermöglicht die Software eine präzise Steuerung und effiziente Ausführung der Aufgaben der Drohne
     
\end{description}

Die Clover-Drohne ist mit einem Raspberry Pi 4 ausgestattet, der die Ausführung von ROS-Programmen erleichtert. Allerdings sollte die Software ressourceneffizient sein, um den begrenzten Ressourcen des Raspberry Pi gerecht zu werden.

\section{Gewünschtes Ergebnis} \label{gewuenschtes_ergebnis:section} 

Das angestrebte Ergebnis des Projekts ist die Entwicklung eines Navigationssystems für die Clover-Drohne, das es ihr ermöglicht, auf der Grundlage eines 3D-Modells der Umgebung genau zu lokalisieren und zu navigieren. Ziel ist es, die erwarteten Herausforderungen wie Sensordrift, Rechenleistung und Kalibrierung zu lösen bzw. alternative Lösungen zu finden.

Das Navigationssystem soll in der Lage sein, die Position der Drohne in Echtzeit zu bestimmen und sie sicher und präzise durch die Umgebung zu führen. Dazu müssen geeignete Lokalisierungstechnologien und Algorithmen eingesetzt werden, die auf den vorhandenen Sensoren der Drohne basieren. Je nach den spezifischen Anforderungen der Anwendung und den Merkmalen der Einsatzumgebung können verschiedene Methoden eingesetzt werden, darunter Trägheitsnavigation, optische Sensoren, Ultraschall- oder Infrarotsensoren, Magnetfeldsensoren oder funkbasierte Lokalisierung.

Neben der Lokalisierung sollte das Navigationssystem auch in der Lage sein, ein 3D-Modell der Umgebung zu erstellen und zu aktualisieren. Dies ermöglicht der Drohne eine präzise Kollisionsvermeidung und eine effiziente Flugplanung. Die \ac{SLAM}-Technologie wird eingesetzt, um Sensordaten zu verarbeiten und ein genaues 3D-Modell zu erstellen. Zu diesem Zweck kann eine geeignete SLAM-Bibliothek wie RTAB-Map, KinectFusion oder ORB-SLAM eingesetzt werden.

Die Softwarekomponenten, einschließlich des PX4-Autopiloten, ROS (Robot Operating System) und SLAM-Algorithmen, sollten effizient implementiert und auf die Ressourcen des Raspberry Pi 4 zugeschnitten sein. Eine geeignete Entwicklungsumgebung und Simulationswerkzeuge werden eine effiziente Softwareentwicklung, Fehlersuche und Überwachung ermöglichen.

Das gewünschte Ergebnis des Projekts ist ein funktionsfähiges Navigationssystem für die Clover-Drohne, das auf der Grundlage eines 3D-Modells der Umgebung genau lokalisieren und navigieren kann. Die entwickelte Software wird auf der Drohne getestet und bewertet, um ihre Leistung und Genauigkeit zu beurteilen. Das Navigationssystem sollte in der Lage sein, verschiedene Flugmanöver und Aufgaben auszuführen, z. B. durch enge Korridore zu navigieren, Hindernissen auszuweichen oder bestimmte Orte in der Umgebung anzuvisieren.


\chapter{Theoretische Grundlagen}


\section{Sensoren}

    \subsection{Magnetometer}

    \subsection{Gyroskyp}

    \subsection{Laserscanner}

    \subsection{Kamera}

        \subsubsection{Bild Kamera}

        \subsubsection{Infrarot Kamera}

    \subsection{Abstandssensoren}

\section{ROS}

    \subsection{Nodes}

    \subsection{Topics}

    \subsection{Publish and Subscripe Pattern}

    \subsection{Objekterkennung}

    \subsection{QR-Codes}

\section{Drohne/Multicopter}
Bei der für das Projekt verwendete Drohne handelt es sich um eine Coex Clover Drohne. Dies ist eine programmierbare Drohne, die besonders für Bildungszwecke eingesetzt wird. Sie ist sowohl für den Einsatz draußen sowie auch in Gebäuden geeignet. \\

\begin{figure}[htpb]
    \centering
    \includegraphics[width=10cm,keepaspectratio,angle=0]{images/coex_clover.jpg}
    \caption[Coex Clover Drohne]{\label{img coex_clover} Coex Clover Drohne \cite{img_coex_clover}}
\end{figure}

Zu Beginn erhält man hierbei einen Bausatz, welcher dann zu einem Quadrokopter zusammengebaut werden kann. Der Vorteil hierbei ist zudem, dass die gesamte Drohne ohne Löten zusammengesetzt werden kann. Zu den einzelnen Bestandteilen der Drohne kommen, noch eine Dokumentation sowie verschiedene Bibliotheken, die es ermöglichen, die Drohne zusammen bauen und fliegen lassen zu können. \\
Die 
Durch die Verwendung verschiedener Open-Source Komponenten lässt sich die Drohne programmieren, wodurch ein vielseitiger Einsatzbereich entsteht.\\

\begin{figure}[htpb]
    \centering
    \includegraphics[width=10cm,keepaspectratio,angle=0]{images/coex_clover_kit.jpg}
    \caption[Bausatz Coex Clover Drohne]{\label{img coex_clover_kit} Bausatz Coex Clover Drohne \cite{img_coex_clover_kit}}
\end{figure}

Die Coex Clover Drohne soll laut Herstellerinformationen bis zu 15 Minuten am Stück fliegen können und in dieser Zeit eine Maximalhöhe von 500 Metern bei einer Höchstgeschwindigkeit von bis zu 72 km/h erreichen können \cite[vgl.][]{coex_clover}.\\

Es gibt verschiedene Versionen der Drohne, bei der hier eingesetzten, handelt es sich um die Coex Clover 4.2.
Wichtige Bestandteile dieser sind hierbei:
\begin{center}
    \begin{itemize}
        \item Flight-Controller Coex Pix
        \item Raspberry Pi 4
        \item 
        \item 
        \item 
        \item 
    \end{itemize}
    \label{lst:coex-components}
\end{center}

Zu den Hauptbestandteilen der Drohne zählen zum einen ein Raspberry Pi 4 sowie der Flightcontroller Coex Pix. Diese bilden die Grundlage zur Programmierung und Steuerung der Coes Clover Drohne und ermöglichen es zudem die Drohne über drahtlos per WLAN zu verbinden. \\
Die Drohne ist ein Quadrokopter und besitzt somit vier Motoren, welche einzelnd angesteuert werden können. Sie besitzt zudem eine Vielzahl verschiedener Sensoren, auf welche in \ref{section:Sensoren} genauer eingegangen wird. Zu diesen zählen unter anderem ein Gyroskop, Magnetometer sowie ein Laseranstandssensor und eine Kamera, die unten an der Drohne angebracht sind.
Zum Schutz befindet sich zudem außen einen Rahmen.



\section{3D-Modelle}

    \subsection{3D-Scanning}

\section{Positionsbestimmung}

    \subsection{Spatial Mapping}

    \subsection{Inertielle Positionsbestimmung}

\section{Regelsysteme}

    \subsection{PID-Regler}

\section{Simulationstechnik}

\section{Problembehebung}

\chapter{Probleme}

\section{Aufgetretene Probleme}

\subsection{Lösungen}


\chapter{Hardware Implementierung}

\section{Komponenten}

\begin{itemize}
    \item{Raspberry Pi 4 1 GB}
    \item{COEX Clover 4.2}
\end{itemize}

\section{COEX Clover 4.2}

\subsection{Abstandssensor}

\subsection{PX4 FlightController}

\begin{itemize}
    \item{Barometer}
    \item{Gyroskop}
    \item{Accelerometer}
    \item{Magnetometer}
\end{itemize}

\subsection{Kamera}

\section{3D Scanner}

\subsection{Microsoft Hololens}

\subsection{Microsoft Kinect}
\chapter{Software Implementierung}

\section{PX4} \label{px4:subsection}
PX 4 ist eine Open-Source-Software, welche zur Steuerung verschiedener Arten von Fahrzeugen genutzt werden kann, hierzu zählen beispielsweise verschiedene Drohnenarten, sowie auch Fahrzeuge auf dem Boden und Unterwasserfahrzeuge.\\ Es kann zum einen für bereits flugfähigen Drohen eingesetzt werden. Aber es besteht auch die Möglichkeit eine neue Drohne in Verbindung mit PX4 zu bauen.\\
Für die Verwendung der PX4 Software kann QGroundControl (siehe Kapitel \ref{qGroundControl:subsection}) verwendet werden. \cite[vgl.][]{px4} \\
Der PX4-Flugstack wurde ursprünglich nur dür die Pixhawk-Hardware entwickelt, allerdings ist es heutzutage auch möglich diesen auf Linux-Computern und anderen Hardware einzusetzen. Wie es auch bei der Coex Clover Drohne mit dem Coex Pix umgesetzt wird. \\
Die Software setzt Sensoren, ein um den Zustand der Drohne zu bestimmen. Hierfür werden einige Sensoren vorausgesetzt, zu diesen zählen ein Gyroskop, ein Beschleunigungssensor, ein Magnetometer sowie ein Barometer. Zudem ist GPS empfohlen um weitere Modi nutzen zu können. \cite[vgl.][]{px4}

\subsection{Flugmodi}
PX4 bietet verschiedene Flugmodi an, welche das Verhalten der des jeweiligen Fahrzeuge beziehungsweise Drohnen steuert und auch regelt, wie jeweils auf Benutzereingaben regiert werden soll. Das Wechseln dieser Flugmodi kann zum einen über die QGroundControl Software (siehe Kapitel \ref{qGroundControl:subsection}) oder auch je nach Anpassung der Fernbedinung, beispielsweise über verschiedene Schalter, auf dieser vollzogen werden. \\
Allerdings muss auch beachtet werden, dass nicht alle Flugmodi bei allen Drohnen oder Fahrzeugen einsetzbar sind. Aussschlaggebend hierfür ist vor allem die Ausstattung des. Da jeder Flugmodi bestimmte Bedingungen hat die erfüllt sein müssen, beispielsweise Sensoren wie ein Geschwindikgeitssensor. \\
Die Flugmodi lassen sich in drei Kategorien einteilen, diese sind manuelle, ünterstütze sowie automatische Steuerung.
Da es bei diesem Projekt um den Einsatz einer Drohne handelt, werden im folgenden lediglich die für Drohnen relevanten Flugmodi erläutert.
Zu der manuellen Steuerung, bei welchem der Pilot die Drohne direkt und ohne direkte UNterstützung steuert, gehören unter anderem die Modi "manuell" beziehungsweise "stabilisiert". Diese sorgen für eine stabilisierte horizontale Ausrichtung, jedoch ermöglichen sie es dem Piloten hierbei, das Gaspedal sowie die Roll- und Neigbewegungen selbst zu bestimmen. Zu dieser Katergorie gibt es noch weitere Modi, welche allerdingsa hauptsächlich für Flugshows genutzt werden.
Bei der unterstützen Steuerung kann aus zwei verschiedenen Flugmodi gewählt werden, "ALTCTL (Altitude)" und "POSCTL (Position)". Beim ALTCTL-Modi wird besonders die Höhe der Drohne von Autopliot gesteurt, sodass diese einen möglichst konstanten Abstand zum Boden hält. Dieser Modi benötigt hierfür ein Barometer oder andere Sensoren zur Höhenmessung.
Der POSCTL dient zum Halten der Position der Drohne, das heißt neben der Höhe werden zudem die Bewegungsgeschwindkeit nach vorne, hinten und zur Seite gesteuert, um die Drohne möglichst auf einer bestimmten Position zu halten.\\
Bei der automatischen Steuerung fliegt die Drohne automatische ohne Benutzereingaben mittels eines Programms. Zu dieser Katergorie gehört der "Offboard"-Modus hierdurch ist es möglich, dass die Drohne durch einen anderen Computer gesteuert werden kann. Zudem gibt es noch einen Missions-Modus, hierbei kann beispielsweise über QGroundControl ein Pfad geplant werden, welcher dann von der Drohne mittels GPS geflogen wird. \\
Für das Projekt ist hierbei vor allem der Offboard-Modus von Bedeutung, da dieser benötigt wird, um autonome Flüge mit der Coex Clover Drohne zu machen, da diese hierbei von dem Raspberry Pi gesteuert wird. \cite[vgl.][]{flight-modes}


\subsection{QGroundControl}  \label{qGroundControl:subsection}
QGroundControl ist eine Software, welche vor allem für Drohnen mit einem PX4 aber auch anderen Flight Controller genutzt werden kann. Hierbei bietet es verschiedene Funktionen. Zum einen gehört hierzu die Konfiguration der einzelnen  Drohnen. Desweiteren ist es möglich mit der Software verschiedene Flugmodi auszuwählen, sowie diese dann auch während des Fluges zu überwachen, beispielweise durch die Anzeige der Flugposition auf einer Karte sowie auch deren Geschwindigkeit und andere Sensordaten.
Es ist auch möglich mit QGroundControl eine ganze Flugplanung zu machen, welche die Drohne daraufhin umsetzt. \cite[vgl.][]{qGroundControl}


\section{Systemarchitektur}

\section{Softwarearchitektur}
\todo{Beschreibung Mavros}

    %\includepdf[landscape=true]{images/graph_ros.pdf}
    %\caption[Übersicht ROS Nodes]{\label{img ros_nodes_graph} Übersicht ROS Nodes [eigene Darstellung]}
    \begin{landscape}
        \begin{figure}
            \includegraphics[width=\paperwidth,keepaspectratio]{images/graph_ros.pdf}
            \caption[Übersicht ROS Kommunikation]{\label{img ros_communication} Übersicht ROS Kommunikation [eigene Darstellung]}
        \end{figure}
    \end{landscape}

In der Abbildung \ref{img ros_communication} ist eine Übersicht der Kommunikation zwischen den verschiedenen Komponenten in \ac{ROS} zu sehen. \\
Hauptbestandteil sind hierbei vor allem die Nodes der Hauptkamera beziehungsweise der Azure Kinect sowie auch Mavros. \\
Mavros ist hierbei ein ROS-Paket, welches die Kommunikation zwischen dem Raspberry Pi und der Drohne mit Hilfe des MAVLink-Protokolls (siehe Kapitel \ref{mavlink}) ermöglicht. Es unterstützt dabei unter anderem den PX4 Flightstack, welcher in der Coex Clover Drohne vorhanden ist. Die Kommunikation kann hierbei wie bei MAVLink auch über \ac{USB} oder auch über Wifi stattfinden. \\
Im nachfolgenden wird nun genauere auf einzelne Komponenten der ROS Kommunikation (siehe \ref{img ros_communication}) eingegangen. \\

%https://clover.coex.tech/en/mavros.html


\section{Kalibrierung der COEX Drohne}

\section{Drohnen Autopilot}


\section{3D Modell}
Um die Drohne auch in der Simulation unter echten Bedingungen fliegen zu lassen, müssen die realen Gegebenheiten in die Simulation übertragen werden. Um dies zu bewerkstelligen, muss ein 3D-Modell der Räume erstellt werden, in welchen sich die Drohne bewegen soll.
    \subsection{3D Scan}

        \subsubsection{Microsoft HoloLens 2}
        Um ein 3D-Modell des Raumes mit der Microsoft HoloLens 2 zu erstellen, muss man die integrierte Mixed-Reality-Capture-Funktion der HoloLens 2 verwenden.
        
        Die Mixed-Reality-Capture-Funktion der HoloLens 2 ist eine integrierte Funktion, mit der Benutzer die Möglichkeit haben, die Hologramme, die von der HoloLens 2 dargestellt werden, in Echtzeit aufzuzeichnen und zu teilen. Diese Funktion ermöglicht es Benutzern, ihre Augenbewegungen und Handlungen in einer virtuellen Umgebung aufzuzeichnen, um sie mit anderen zu teilen oder für zukünftige Referenz oder Analyse zu speichern. Die Mixed-Reality-Capture-Funktion der HoloLens 2 kann auch für die Erstellung von Immersive-Mixed-Reality-Videoinhalten verwendet werden, indem sie es ermöglicht, virtuelle Hologramme in eine reale Umgebung zu integrieren. Dies kann für verschiedene Anwendungen wie zum Beispiel für die Erstellung von virtuellen Touren, für die Präsentation von Produkten oder für die Unterhaltungsindustrie eingesetzt werden.
        
        \todo{Erstellen des Modells beschreiben}

        Es ist wichtig zu beachten, dass die Qualität des erstellten 3D-Modells von verschiedenen Faktoren abhängt, wie z.B. der Beleuchtung im Raum und der Genauigkeit des Scans. Eine sorgfältige Vorbereitung des Raums und eine langsame, gründliche Durchführung des Scans können dazu beitragen, ein genaueres und detaillierteres 3D-Modell zu erstellen.
        \subsubsection{Azure Kinect \ac{DK}}

    \subsection{3D Modell Vorbereitung}

\section{Navigation}

\section{Simulation}




\include{chapter7_fazit}

% Ab hier beginnt der Anhang
\appendix
\addcontentsline{toc}{chapter}{Anhang}

\addcontentsline{toc}{chapter}{Index}
\printindex

\addcontentsline{toc}{chapter}{Literaturverzeichnis}

% Haben Sie das "biblatex"-Paket nicht installiert, benutzen Sie folgendes:
% Ohne das "biblatex"-Paket (s. bericht.sty) produziert folgendes
% "deutsche" Zitate in Literaturverzeichnissen gemaß der Norm DIN 1505,
% Teil 2 vom Jan. 1984.
% Die Zitatmarken werden alphabetisch nach Verfassern
% sortiert und sind durch abgekürzte Verfasserbuchstaben plus
% Erscheinungsjahr in eckigen Klammern gekennzeichnet.

% \bibliographystyle{alphadin}
% \bibliography{bericht}

%%%%%%%%%%%%%%%%%%%%%%%%%%%%%%%%%%%%%%%5
% BIBLATEX
% Benutzt man das "biblatex"-Paket, muß man folgendes schreiben:
\def\refname{Literaturverzeichnis}
\printbibliography
%%%%%%%%%%%%%%%%%%%%%%%%%%%%%%%%%%%%%%%5


%%%%%%%%%%%%%%%%%%%%%%%%%%%%%%%%%%%%%%%%%%%%%%%%%%%%%%%%%%%%%%%%%%%%%%%%%%%%%%%
%% Descr:       Vorlage für Berichte der DHBW-Karlsruhe, Änderungshistorie
%% Author:      Prof. Dr. Jürgen Vollmer, vollmer@dhbw-karlsruhe.de
%% $Id: changelog.tex,v 1.12 2016/03/16 16:56:31 vollmer Exp $
%% -*- coding: utf-8 -*-
%%%%%%%%%%%%%%%%%%%%%%%%%%%%%%%%%%%%%%%%%%%%%%%%%%%%%%%%%%%%%%%%%%%%%%%%%%%%%%%

\chapter*{Änderungen}

\begin{description}
\item[2016/03/16] AUf UTF-8 umgestellt, Indices.
\item[2010/04/12] ToDo-Markierungen mit dem \verb+\todo+-Kommando.
\item[2010/01/27] Anhang (\texttt{appendix}), Selbständigkeits-Erklärung, \texttt{framed}-Paket.
\item[2010/01/21] Abkürzungen (\texttt{acronym}), \texttt{table} und \texttt{tabular} benutzt,
     unübliche Pakete beigelegt.
\item[2010/01/18] Code-Listings (\texttt{listings}), Literaturreferenzen \texttt{biblatex})
\item[2010/01/11] Initiale Version.
\end{description}


\newpage
\addcontentsline{toc}{chapter}{Liste der ToDo's}
\listoftodos[Liste der ToDo's]


\end{document}
