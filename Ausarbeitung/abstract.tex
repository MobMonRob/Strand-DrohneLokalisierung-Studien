\renewcommand{\abstractname}{}

\begin{abstract} % 200 bis 250 Wörter

\section*{Kurzfassung}
\textbf{3D Navigation anhand eines externen Umweltmodels}

Die Navigation einer Drohne anhand eines externen Umweltmodells befasst sich damit, Drohnen im Innenraum eines Gebäudes zu navigieren. \\
Es ist wichtig sich mit dem Thema zu befassen, da es innerhalb eines Gebäudes nicht möglich ist Drohnen mittels \ac{GPS} zu navigieren, daher müssen alternative Navigationsmethoden sowie geeignete Möglichkeiten für die Erstellung eines externen Umweltmodels gefunden werden. \\
Die vorliegende Arbeit handelt von der Umsetzung der Navigation von Drohnen innerhalb eines Gebäudes. Hierfür wird eine bereits vorhandene Drohne flugtauglich gemacht, sodass diese für die Umsetzung des Projektes eingesetzt werden kann. Zudem werden verschiedene Optionen verglichen, woraufhin die geeingnetste Methode mit Hilfe der Drohne umgesetzt wird. \\
Ziel dieser Arbeit ist es, die vorhandene Drohne mittels eines externen Umweltmodels innerhalb eines Gebäudes navigieren zu können. Dadurch soll die Drohne in geschlossenen Räumen autonom fliegen und dabei kleinere Aufgaben wie beispielsweise das Scannen eines QR-Codes ausführen können.

\todo{Abstract überarbeiten am Ende}
\todo{Abstract englisch machen}

\end{abstract}
\selectlanguage{english}
\renewcommand{\abstractname}{}
\begin{abstract}

\section*{Abstract}
\textbf{3D navigation using an external environmental model}

\end{abstract}
\selectlanguage{german}