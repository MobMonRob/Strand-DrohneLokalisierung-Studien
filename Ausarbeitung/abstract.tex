\renewcommand{\abstractname}{}

\begin{abstract} % 200 bis 250 Wörter

\section*{Kurzfassung}
\textbf{3D Navigation anhand eines externen Umweltmodels}

Die Navigation einer Drohne anhand eines externen Umweltmodells befasst sich damit, Drohnen im Innenraum eines Gebäudes zu navigieren. \\
Es ist wichtig sich mit diesem Thema zu befassen, da es nicht möglich ist, Drohnen innerhalb eines Gebäudes mittels \ac{GPS} zu navigieren, daher müssen alternative Navigationsmethoden sowie geeignete Möglichkeiten  zur Erstellung eines externen Umweltmodells gefunden werden. \\
Die vorliegende Arbeit handelt von der Umsetzung der Navigation von Drohnen innerhalb eines Gebäudes. Hierfür wird eine bereits vorhandene Drohne flugtauglich gemacht, sodass diese für die Umsetzung des Projektes eingesetzt werden kann. Zudem werden verschiedene Optionen verglichen, woraufhin die am besten geeignete Methode umgesetzt wird. \\
Ziel dieser Arbeit ist es, die vorhandene Drohne mittels eines externen Umweltmodells innerhalb eines Gebäudes navigieren zu können. Dadurch soll die Drohne in der Lage sein in geschlossenen Räumen autonom zu fliegen und kleinere Aufgaben wie beispielsweise das Scannen eines QR-Codes durchführen zu können.

\todo{Abstract überarbeiten am Ende}
\todo{Abstract englisch machen}

\end{abstract}
\selectlanguage{english}
\renewcommand{\abstractname}{}
\begin{abstract}

\section*{Abstract}
\textbf{3D navigation using an external environmental model}

\end{abstract}
\selectlanguage{german}