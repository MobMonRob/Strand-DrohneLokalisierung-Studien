\chapter{Eigener Ansatz}

\section{Zielsetzung}
Die vorliegende Arbeit handelt davon mit Hilfe eines eigens erstellten 3D Modells eine Drohne Indoor navigieren zu können. Um dies umsetzen zu können gibt es einen Quadrokopter der zu diesem Zwecke eingesetzt werden kann. \\
Heutzutage basiert die Navgiation vieler Drohnen auf \ac{GPS} allerdings kann man Drohnen innerhalb von Gebäuden nicht mit dieser Technik fliegen lassen, da damit die im Innenraum vorhandenen Hindernisse nicht berücksichtigt werden können. Aus diesem Grund soll hierfür eine Möglichkeit gefunden werden, welche es ermöglicht eine Drohne ohne \ac{GPS} im Innenraum eines Gebäudes mit Hilfe eines 3D Modells navigeren zu können.\\
Der Quadrokopter, die Coex Clover Drohne ist bereits mit verschiedenen Sensoren ausgestattet und kann potientiell mit weiteren Sensoren erweitert werden. Jedoch ist die Drohne bis zum Beginn dieser Arbeit noch nicht richtig geflogen ist. Somit muss zum einen die Nutzbarkeit der Drohne für dieses Projekt sichergestellt werden, um diese in dieser Arbeit einsetzen zu können. \\
Ziel dieser Arbeit ist es, bestenfalls den bereits vorhandenen Quadrokopter mit Hilfe eines eigens erstellten 3D Umweltmodells innerhalb eines Raumes navigieren zu können und dadurch kleinere Aufgaben wie beispielsweise das Auslesen eines QR-Codes somit erledigen zu können.



\section{Vorgehensweise}
Die Vorgehensweise ist in der folgenden Abbildung dargstellt:
\todo{Abbildung Vorgehenweise?}


