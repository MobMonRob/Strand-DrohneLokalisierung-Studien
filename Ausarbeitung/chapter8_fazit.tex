\chapter{Fazit}
Zu Beginn der Arbeit wurde sich zunächst mit der COEX Clover Drohne auseinandergesetzt. Hierbei wurde schnell deutlich, dass die Drohne in dem damaligen Zustand nicht Flugtauglich war. Aufgrund einer schlechten Ansteuerung der Motoren durch die \ac{PID} Regler, sowie einen defekten Motor vibrierte die Drohne so stark, dass sie nicht zu kontrollieren war. Die Fehlersuche und dass anschließende Lösen der Probleme hatte allerdings deutlich länger gedauert, als anfangs angenommen. Die fehlerhafte Ansteuerung der Motoren konnte letztendlich beseitigt werden. Des Weiteren wurde der defekte Motor ausgetauscht. Allerdings konnte der Sensordrift der Drohne nicht eliminiert werden.

Zudem wurden nach verschiedene Optionen gesucht, welche eine Erstellung des Umweltmodells für die Navigation ermöglichen. Anfänglich wurde hierfür die Microsoft Hololens 2 verwendet. Das von ihr erstellte Modell muss allerdings vor dem eigentlichen Flug der Drohne aufgenommen werden. Eine dynamische Erstellung des Modells zur Flugzeit der Drohne ist mit der Hololens nicht möglich. Als alternative wurde dann im weiteren Verlauf auf die Azure Kinect für die Erstellung des Umweltmodells
Gewechselt. Die Azure Kinect bietet die Möglichkeit, das Modell in Echtzeit zu erstellen, sowie die eigene Position im erstellten Modell zu ermitteln. Um dies zu bewerkstelligen wird der SLAM-Algorithmus verwendet. Nach einer weitreichenden Analyse der möglichen Algorithmen hat man sich für den ORB-SLAM3 Algorithmus entschieden.

Für die Kommunikation zwischen den einzelnen Komponenten (z.B. Azure Kinect und COEX Clover Drohne) sowie die Steuerung der Drohne und die visuelle Darstellung der Komponentenwerte (z.B. Kamerabild) wurde \ac{ROS} verwendet.

Da man im Laufe der Arbeit unsicher war, ob man die COEX Clover Drohne, aufgrund der oben benannten Probleme, letztendlich überhaupt benutzen kann entschloss man sich dazu, neben der realen Drohne eine Simulation zu verwenden. Mit dem Simulator Gazebo konnte der autonome Flug der Drohne innerhalb eines, von der Hololens vorher erstellten Umweltmodells umgesetzt werden. Auch hierbei wurde Kommunikation mit der simulierten Drohne \ac{ROS} verwenden.


Schlussendlich konnte die Navigation der Coex CLover Drohne zusammen mit der Azure Kinect allerdings nicht in der realen Welt getestet werden. Die SLAM-Algorithmen sind sehr rechenintensiv. Der von der Drohne verwendete Raspberry Pi 4 hat nicht genügend Kapazitäten, um den SLAM-Algorithmus zur Flugzeit der Drohne anzuwenden. Somit ist eine dynamische Erstellung des 3D-Modells der Umgebung sowie die Positionsfindung innerhalb dieses Modells mit den gegebenen Komponenten nicht möglich.



\section{Ausblick}
Um die Drohne auch letztendlich sicher in Innenräumen fliegen lassen zu können ist es wichtig, den bestehenden Sensordrift so gut wie möglich zu minimieren. Es ist wichtig, die Ursachen des Sensordrifts zu ermitteln und zu verstehen. Dies kann eine gründliche Analyse der verwendeten Sensoren und ihrer Genauigkeit sowie der Umweltfaktoren, die zum Drift beitragen könnten, erfordern. Auf der Grundlage dieser Erkenntnisse können geeignete Korrekturalgorithmen entwickelt oder angewendet werden, um die Messungen zu verbessern und die Auswirkungen des Drifts zu minimieren. Zudem kann man SLAM verwenden, um eine Fusion der Sensordaten durchzuführen um somit die fehlerbehafteten Daten herausgerechnet werden. 

Um SLAM auch zur Flugzeit der Drohne verwenden zu können muss die Azure Kinect an der Drohne befästigt werden.
Um den SLAM-Algorithmus in Echtzeit nutzen zu können, ist eine leistungsstärkere Recheneinheit erforderlich. Es können Alternative in Betracht gezogen werden, die eine höhere Rechenleistung bieten, um die Echtzeitverarbeitung der Sensordaten zu ermöglichen. Eine gründliche Evaluation der verfügbaren Optionen könnte dabei helfen, die beste Lösung für die gegebenen Anforderungen zu finden. Des Weiteren kann nach alternativen SLAM Verfahren, welche weniger rechenaufwändig sind, gesucht werden, damit man den Raspberry Pi nicht ersetzten muss. Ein weiterer Punkt zu Lösung des Problems wäre es, Optimierungen für ORB-SLAM3 zu finden, welche ebenfalls dafür sorgen, dass der Algorithmus weniger rechenaufwändig wird. 

Die Umsetzung der Pfadfindung mit Hinderniserkennung und -vermeidung auf der Grundlage von SLAM erfordert die Integration von Echtzeit-Algorithmen zur Hinderniserkennung und -vermeidung. Es müssen Strategien entwickelt werden, um sicher um Hindernisse herum zu navigieren und alternative Wege zu planen. Außerdem müssen verschiedene Algorithmen zur Wegfindung verglichen werden, um den passenden zu implementieren.