\chapter{Fazit}
Zu Beginn der Arbeit wurden zunächst verschiedene Optionen gesucht, mit welchen eine Erstellung des Umweltmodells für Navigation ermöglichen. Anfänglich wurde mit Hilfe der Microsoft HoloLens 2 die Erstellung eines 3D Modells genutzt, jedoch musste hiermit das Modell vor dem eigentlichen Flug der Drohne aufgenommen werden. \\
Da für das Projekt eine gut funktionierende Drohne nötig ist, musste hier zuerst Anpassungen an der Coex Clover Drohne gemacht werden, sodass diese auch dafür eingesetzt werden konnte. Allerdings konnten hier nicht alle Probleme beseitigt werden.\\ 
Desweiteren konnte mit Hilfe des Simulators Gazebo der autonome Flug der Coex Drohne innerhalb eines vorher erstellten Umweltmodells umgesetzt werden. \\
Im weiteren Verlauf wurde auch die Azure Kinect für die Erstellung des Umweltmodells gewechselt, da bei dieser die Möglichkeit besteht das Modell zu erstellen sowie die eigene Position in Echtzeit mit Hilfe des SLAM-Algorithmus zu ermitteln, zudem stellte sich die Befestigung der HoloLens als Problem heraus. \\ 
Schlussendlich konnte die Navigation der Coex CLover Drohne zusammen mit der Azure Kinect allerdings nicht in der realen Welt getestet werden. Ursache hierfür war unter anderem zu geringe Leistung der Raspberry Pi.
\todo{muss noch erweitert werden}



\section{Ausblick}
Um die Drohne auch letztendlich sicher in Innenräumen fliegen lassen zu können ist es wichtig, den bestehenden Sensordrift so gut wie möglich zu minimieren. Es ist wichtig, die Ursachen des Sensordrifts zu ermitteln und zu verstehen. Dies kann eine gründliche Analyse der verwendeten Sensoren und ihrer Genauigkeit sowie der Umweltfaktoren, die zum Drift beitragen könnten, erfordern. Auf der Grundlage dieser Erkenntnisse können geeignete Korrekturalgorithmen entwickelt oder angewendet werden, um die Messungen zu verbessern und die Auswirkungen des Drifts zu minimieren. Zudem kann man SLAM verwenden, um eine Fusion der Sensordaten durchzuführen um somit die fehlerbehafteten Daten herausgerechnet werden. 

Um SLAM auch zur Flugzeit der Drohne verwenden zu können muss die Azure Kinect an der Drohne befästigt werden.
Um den SLAM-Algorithmus in Echtzeit nutzen zu können, ist eine leistungsstärkere Recheneinheit erforderlich. Es können Alternative in Betracht gezogen werden, die eine höhere Rechenleistung bieten, um die Echtzeitverarbeitung der Sensordaten zu ermöglichen. Eine gründliche Evaluation der verfügbaren Optionen könnte dabei helfen, die beste Lösung für die gegebenen Anforderungen zu finden. Des Weiteren kann nach alternativen SLAM Verfahren, welche weniger rechenaufwändig sind, gesucht werden, damit man den Raspberry Pi nicht ersetzten muss. Ein weiterer Punkt zu Lösung des Problems wäre es, Optimierungen für ORB-SLAM3 zu finden, welche ebenfalls dafür sorgen, dass der Algorithmus weniger rechenaufwändig wird. 

Die Umsetzung der Pfadfindung mit Hinderniserkennung und -vermeidung auf der Grundlage von SLAM erfordert die Integration von Echtzeit-Algorithmen zur Hinderniserkennung und -vermeidung. Es müssen Strategien entwickelt werden, um sicher um Hindernisse herum zu navigieren und alternative Wege zu planen. Außerdem müssen verschiedene Algorithmen zur Wegfindung verglichen werden, um den passenden zu implementieren.